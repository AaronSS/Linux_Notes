\documentclass{beamer}
\usepackage[adobefonts,UTF8]{ctexcap}
\usepackage{stmaryrd}
\usepackage{amsmath}
%\usepackage{multicol}

\title{Linux Install Party}
%\author{}
%\date{\today}
\begin{document}

\begin{frame}
\titlepage
\end{frame}
\begin{frame}
%\begin{multicols}{2}
\tableofcontents[part=1]
%\end{multicols}
\end{frame}

\begin{frame}
\tableofcontents[part=2]
\end{frame}

\part{1}
\section{准备工作}
\begin{frame}{如何获取Linux}
各Linux发行版的安装镜像都能在网上找到。为了便于发行,很多发行版在世界各地都有镜像源,镜像源中包含了该发行版的软件包和安装镜像。
在教育网中,建议从镜像源下载Linux的安装镜像,常见的镜像源有:
\begin{itemize}
\item \url{http://mirrors.tuna.tsinghua.edu.cn}
\item \url{http://mirrors.ustc.edu.cn}
\item \url{http://mirrors.163.com}
\end{itemize}
\end{frame}

\begin{frame}{单系统,多系统,还是虚拟机?}
	\begin{itemize}
		\item	几乎所有能用计算机做的工作都可以在Linux下完成,因此,如果没特殊需求,可以直接用Linux单系统。
		\item 不过还是有不少人选择保留原有的系统(Windows,Mac,不同的Linux发行版),这时可以让多系统共存。
		\item 使用虚拟机,可以让用户同时使用多个系统。可以在Linux下用虚拟机安装其他系统,
			也可以在其他系统用虚拟机安装Linux.
	\end{itemize}
\end{frame}

\begin{frame}
\section{选择一个发行版}
\subsection{我是新手}
新手对Linux中的很多概念都不了解,有的甚至是刚接触电脑,这时应该用一个适合新手的发行版。
\begin{itemize}
\item Ubuntu: 很流行,但是经常被黑,主要问题是和上游关系不好,Unity桌面等。
\item Fedora: 小白鼠发行版,常常瞻前不顾后,有的软件会出些问题。
\item Redhat 9: 很多老Linux教程中提到的发行版,已经过时。
\end{itemize}

因此,我推荐的新手发行版是\ldots
\end{frame}

\begin{frame}{推荐新手使用的发行版}
\begin{itemize}
\item Deepin: 基于Ubuntu的国产发行版,非常适合国人使用,有独立开发的桌面环境,搜狗拼音输入法,QQ,迅雷等软件。
\item openSUSE: 华丽的发行版,有最好的KDE桌面,系统配置工具强大,不错的文档和社区支持。
\item Linux Mint: Debian系的发行版,常年排名靠前,有传统而又现代的Cinnamon桌面。
\end{itemize}
\end{frame}

\subsection{中级用户}
\begin{frame}{中级用户}
如果已经有了一定的水平,想用一个定制性稍高的发行版,那么可以用下列发行版。
\begin{itemize}
\item Debian: 老牌发行版,稳定,拥有强大的包管理和配置工具。
\item Arch: 崇尚极简主义,前卫的发行版,强大的文档和社区支持。
\item Gentoo: 超高定制性的基于源码的发行版,拥有最强大的包管理器。
\end{itemize}

\end{frame}

\subsection{高级用户}
\begin{frame}{高级用户}
如果你是高级用户,已经非常熟悉Linux,那就没必要来这里选发行版了\ldots
\end{frame}

\part{2}
\section{安装方法}
\begin{frame}{安装Linux}
很多发行版都有一个图形化的安装程序,只要按照向导安装就能装好系统。
我们这里要更深入地讨论系统的安装,了解其中的关键流程。
\end{frame}

\subsection{Debian,Arch,Gentoo的安装}
\begin{frame}{使用debootstrap安装Debian}
debootstrap可以用于创建一个基础的Debian系统,在很多发行版中都有此工具。用它安装Debian的大致流程如下:
\begin{enumerate}
\item 分区并挂载。
\item 用debootstrap在目标根目录下做一个基础系统。
\item chroot进去用apt安装内核和引导器(bootloader)。
\item 配置fstab,设置root密码,把bootloader安装到磁盘的引导区,配置bootloader。
\item 退出chroot环境,重新启动。
\end{enumerate}
\end{frame}

\begin{frame}{安装Arch}
Arch的LiveCD中有一个install.txt,详细地讲解了安装方法。
\begin{enumerate}
\item 仍然是分区,挂载。
\item 用pacstrap把base组的包全部装到目标根文件系统,其中包含基本工具和内核。
\item 用pacstrap安装bootloader.
\item 利用arch-chroot把bootloader安装到引导区,配置bootloader,设置root密码。
\item 利用genfstab生成fstab.
\item 重新启动。
\end{enumerate}
\end{frame}

\begin{frame}{安装Gentoo}
按照Gentoo Handbook,Gentoo的安装方法大致如下:
\begin{enumerate}
\item 仍然是分区,挂载。
\item 下载一份stage3,解压至目标根文件系统。
\item 下载一份portage树,解压至目标的/usr下,并用emerge --sync同步portage树至最新。
\item chroot进入目标系统。
\item emerge一份内核源码,配置,编译,安装内核。
\item emerge一份bootloader,并将其安装至引导区,配置bootloader。
\item emerge其他常用的工具,用于连网,记录日志,执行计划任务等。
\item 退出chroot并重启。
\end{enumerate}
\end{frame}

\begin{frame}{要点}
从Debian,Arch,Gentoo的安装过程可以看出一些共同点:
\begin{itemize}
\item 都构建了一个基本系统(debootstrap,pacstrap,stage3).
\item 使用了一个叫做chroot的手段。
\end{itemize}

什么叫做chroot(change root)?简单地说,就是让用户进入一个环境,其根目录为一个指定目录。
\end{frame}

\subsection{总结}
\begin{frame}{安装Linux系统的关键是\ldots}
\Huge 基本系统 + chroot \normalsize
\end{frame}

\subsection{常见问题}
\begin{frame}{常见问题}
\begin{itemize}
\item chroot前请先挂载/proc,/dev,/sys
\item 别忘了设置root密码及配置fstab
\item 重启前请确认必要的包都装了,最好先配置好网络
\end{itemize}
\end{frame}

\section{LiveUSB的制作}
\begin{frame}{LiveUSB的制作}
现在用光盘安装系统已经不流行了,对于可以自由下载的Linux发行版更是如此。以下介绍LiveUSB的制作方法。
\end{frame}

\subsection{直接写盘法}
\begin{frame}{直接写盘法}
这种方法是直接把ISO镜像写入U盘,然后电脑启动时会把U盘作为一个光盘启动。
\begin{itemize}
\item 常用工具为dd,简单粗暴。
\item 写入ISO镜像后,U盘基本上就没别的用途了。
\item 效果不好。
\end{itemize}

因此本人不推荐这种做法。
\end{frame}

\subsection{辅助工具法}
\begin{frame}{辅助工具法}
有一些工具可以利用LiveCD的ISO制作LiveUSB,效果还不错。
常用的有UnetBootin, Universal USB installer, Linux Live USB Creater等,
这类工具的使用很简单,按照工具的图形界面的提示制作就行了。
\begin{itemize}
\item 方法简单。
\item 一般来说U盘无需格式化。
\item 只能把一个ISO制作成LiveUSB,似乎也有多重启动U盘的制作工具。
\end{itemize}
\end{frame}

\subsection{手工制作}
\begin{frame}{手工制作}
不借助LiveUSB制作工具,直接手工操作,可以打造出一个高度定制的LiveUSB.
我们需要的,是一个引导器,常用的有syslinux,grub4dos,grub2

ArchWiki上有手工制作Arch的LiveUSB的方法:
\begin{enumerate}
\item 把LiveCD里面的所有文件复制到U盘
\item 安装syslinux到U盘,并给U盘的分区设定可引导标志
\item 修改syslinux配置文件
\end{enumerate}

\end{frame}

\begin{frame}{制作LiveUSB的一般方法}
手工制作LiveUSB的要点:
\begin{itemize}
\item 安装引导器
\item 复制必要的文件
\item 编写/修改配置文件
\end{itemize}

提示:
\begin{itemize}
\item 最重要的是要学会模仿,主要是iso镜像自身和各种工具生成的syslinux.cfg
\item memdisk等小工具使用于小镜像
\item 如果制作的LiveUSB出了问题,还是要看看相关文档
\end{itemize}

\end{frame}

\section{其他}
\begin{frame}{其他}
一些安装中遇到的问题:
\begin{itemize}
\item 多系统共存
\item UEFI和GPT分区表
\item 虚拟机的使用
\end{itemize}

我们将在剩下的时间解决这些问题。
\end{frame}

\end{document} 
